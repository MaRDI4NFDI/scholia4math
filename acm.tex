\documentclass[sigconf,natbib=false]{acmart}

%%
%% \BibTeX command to typeset BibTeX logo in the docs
\AtBeginDocument{%
  \providecommand\BibTeX{{%
    \normalfont B\kern-0.5em{\scshape i\kern-0.25em b}\kern-0.8em\TeX}}}

%% Rights management information.  This information is sent to you
%% when you complete the rights form.  These commands have SAMPLE
%% values in them; it is your responsibility as an author to replace
%% the commands and values with those provided to you when you
%% complete the rights form.
\setcopyright{acmcopyright}
\copyrightyear{2018}
\acmYear{2018}
\acmDOI{10.1145/1122445.1122456}

%% These commands are for a PROCEEDINGS abstract or paper.
\acmConference[Woodstock '18]{Woodstock '18: ACM Symposium on Neural
  Gaze Detection}{June 03--05, 2018}{Woodstock, NY}
\acmBooktitle{Woodstock '18: ACM Symposium on Neural Gaze Detection,
  June 03--05, 2018, Woodstock, NY}
\acmPrice{15.00}
\acmISBN{978-1-4503-XXXX-X/18/06}


%%
%% Submission ID.
%% Use this when submitting an article to a sponsored event. You'll
%% receive a unique submission ID from the organizers
%% of the event, and this ID should be used as the parameter to this command.
%%\acmSubmissionID{123-A56-BU3}

%%
%% The majority of ACM publications use numbered citations and
%% references.  The command \citestyle{authoryear} switches to the
%% "author year" style.
%%
%% If you are preparing content for an event
%% sponsored by ACM SIGGRAPH, you must use the "author year" style of
%% citations and references.
%% Uncommenting
%% the next command will enable that style.
%%\citestyle{acmauthoryear}
%--
\usepackage[
backend=biber,
style=numeric,
firstinits=true,
url=false,
isbn=false,
safeinputenc,%required for arxiv
]{biblatex}
%Add the line below and make sure to not change the names
%Also copy the file biber.conf which uses the names
\addbibresource{short.bib}
% Add additional usepackage commands here.
% If you intent to submit to an ACM conference please consult
% https://www.acm.org/publications/taps/whitelist-of-latex-packages
% first.

%TODO: Change the title below
\title{Discovering Mathematical Objects of Interest---A Study of Mathematical Notations}

%%
%% end of the preamble, start of the body of the document source.
\begin{document}

%%
%% The "title" command has an optional parameter,
%% allowing the author to define a "short title" to be used in page headers.

%%
%% The "author" command and its associated commands are used to define
%% the authors and their affiliations.
%% Of note is the shared affiliation of the first two authors, and the
%% "authornote" and "authornotemark" commands
%% used to denote shared contribution to the research.
\newcommand{\aff}[1]{\texorpdfstring{$^{#1}$}{}}
\author{Moritz Schubotz\aff{1,2}, Andr\'{e} Greiner-Petter\aff{1},
Norman Meuschke\aff{1,3}, Bela Gipp\aff{1,3}}
%\authornote{Correspondent author.}
\affiliation{%
  $^1$ University of Wuppertal, Germany ({andre.greiner-petter@zbmath.org, \{last\}@uni-wuppertal.de})\\
  $^3$ University of Konstanz, Germany (\{first.last\}@uni-konstanz.de)
%    \institution{University of Wuppertal}
%    \city{Wuppertal}%
%    \country{Germany}
}
%\email{andre.greiner-petter@zbmath.org}

%\author{Moritz Schubotz,}
%%\email{schubotz@uni-wuppertal.de}
%%\affiliation{%
%%    \institution{University of Wuppertal}%
%%    \city{Wuppertal}%
%%    \country{Germany}
%%}
%\author{Fabian M\"{u}ller}
%\affiliation{%
%    \institution{FIZ Karlsruhe / zbMATH}%
%    \city{Berlin}%
%    \country{Germany}
%}
%\email{{moritz,fabian}@zbmath.org}
%
%\author{Corinna Breitinger}
%\affiliation{%
%    \institution{University of Konstanz}%
%    \city{Konstanz}%
%    \country{Germany}
%}
%\email{corinna.breitinger@uni-konstanz.de}
%
%\author{Howard S.~Cohl}
%\affiliation{%
%    \institution{%
%    	National Institute of Standards and Technology}%
%    \country{USA}
%}
%\email{howard.cohl@nist.gov}
%
%\author{Akiko Aizawa}
%\affiliation{%
%    \institution{National Institute of Informatics}%
%    \city{Tokyo}%
%    \country{Japan}
%}
%\email{aizawa@nii.ac.jp}
%
%\author{Bela Gipp}
%\affiliation{%
%    \institution{University of Wuppertal}%
%    \city{Wuppertal}%
%    \country{Germany}
%}
%\email{gipp@uni-wuppertal.de}

\renewcommand{\shortauthors}{Schubotz, et al.}

%%
%% The abstract is a short summary of the work to be presented in the
%% article.
\begin{abstract}
% Write the abstract below.
% Start each sentence with a new line.
% Avoid special LaTeX commands and environments as much as possible.
\end{abstract}

%%
%% The code below is generated by the tool at http://dl.acm.org/ccs.cfm.
%% Please copy and paste the code instead of the example below.
%%
\begin{CCSXML}
<ccs2012>
 <concept>
  <concept_id>10010520.10010553.10010562</concept_id>
  <concept_desc>Computer systems organization~Embedded systems</concept_desc>
  <concept_significance>500</concept_significance>
 </concept>
 <concept>
  <concept_id>10010520.10010575.10010755</concept_id>
  <concept_desc>Computer systems organization~Redundancy</concept_desc>
  <concept_significance>300</concept_significance>
 </concept>
 <concept>
  <concept_id>10010520.10010553.10010554</concept_id>
  <concept_desc>Computer systems organization~Robotics</concept_desc>
  <concept_significance>100</concept_significance>
 </concept>
 <concept>
  <concept_id>10003033.10003083.10003095</concept_id>
  <concept_desc>Networks~Network reliability</concept_desc>
  <concept_significance>100</concept_significance>
 </concept>
</ccs2012>
\end{CCSXML}

\ccsdesc[500]{Computer systems organization~Embedded systems}
\ccsdesc[300]{Computer systems organization~Redundancy}
\ccsdesc{Computer systems organization~Robotics}
\ccsdesc[100]{Networks~Network reliability}

%%
%% Keywords. The author(s) should pick words that accurately describe
%% the work being presented. Separate the keywords with commas.
\keywords{datasets, neural networks, gaze detection, text tagging}

%% A "teaser" image appears between the author and affiliation
%% information and the body of the document, and typically spans the
%% page.
%\begin{teaserfigure}
%  \includegraphics[width=\textwidth]{sampleteaser}
%  \caption{Seattle Mariners at Spring Training, 2010.}
%  \Description{Enjoying the baseball game from the third-base
%  seats. Ichiro Suzuki preparing to bat.}
%  \label{fig:teaser}
%\end{teaserfigure}

%%
%% This command processes the author and affiliation and title
%% information and builds the first part of the formatted document.
\maketitle
\section{Introduction}\label{sec:intro}

\section{Method}\label{sec:method}

\section{Evaluation}\label{sec:eval}

\section{Conclusion \& Future Work}\label{sec.concl}

\end{document}
\endinput
%%
%% End of file .
